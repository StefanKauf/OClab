\section*{Fragestellung und Methoden}
\begin{itemize}
    \item Beschreiben Sie in klaren S\"atzen den Aufbau und die Funktion der einzelnen Elemente. 
    \item Unterst\"utzen Sie ihr Ausf\"uhrungen - wenn sinnvoll - mit einer bildlichen Darstellung. Dabei gelten folgende Grundregeln:
    \begin{itemize}
        \item Sie m\"ussen auf jede Darstellung im Text Bezug nehmen und diese erkl\"aren. Den Lesenden muss klar sein was das Bild darstellt und wozu es im Protokoll verwendet wird.
        \item Geben Sie die benutzen Quellen an. Bei eigenen Darstellungen muss keine Angabe erfolgen.
        \item Diagramme m\"ussen vollst\"andig beschriftet sein, das hei"st: Titel, Achsenbeschriftungen (Physikalische Gr\"o"se und Einheit) und Legende. Gleichzeitig m\"ussen die Diagramme gut lesbar sein und \"uber eine sinnvolle Skalierung verf\"ugen. 
        \item Jede Darstellung verf\"ugt \"uber eine Bildunterschrift und eine Nummerierung. 
     \end{itemize}
     \item Geben Sie Gleichung wie folgt an
     \begin{align}
   			 F_t(P) & = (M_t.\nabla)B_t(P)
   		 	\label{eq:equation1}
	\end{align}
 	und erklären Sie jedes Formelzeichen im Text.
 	\item Ableitungen höherer Ordnung werden wie folgt angeschrieben:
 	\begin{equation}
 		x, \dot{x}, \ddot{x}, x^{\left(3\right)}, x^{\left(4\right)}, \dots
 	\end{equation}
 	\item Mögliche Unterkapitel sind:
 	\begin{itemize}
 		\item Theoretische Einführung
 		\item Versuchsaufbau
 	\end{itemize}
\end{itemize}