\section*{Wissenschaftliches Schreiben}

\begin{quotation}
''Unter einem Plagiat ist die ganze oder teilweise Übernahme eines fremden oder eigenen Werks ohne Angabe der Quelle und des Urhebers bzw. der Urheberin zu verstehen. Das Plagiat geht oft mit einer Verletzung (Fremdplagiat) oder einem Missbrauch (Selbstplagiat) des Urheberrechts einher. Allerdings ist das Plagiat als wissenschaftliches Fehlverhalten unabhängig von und nicht deckungsgleich mit urheberrechtlichen Tatbeständen. Kürzere Passagen eines fremden Werkes dürfen zitiert werden. Dies setzt aber eine Kennzeichnung des Zitats und eine Angabe der Quelle voraus.'' \citep{Umit_plag}
\end{quotation}
So sehen es die Plagiatsrichtlinien der UMIT vor. Die aktuell gültige Version kann auf der Website heruntergeladen werden.
\newline

Zitate in wissenschaftlichen Arbeiten dienen dazu, transparent anzugeben woher die angegebenen Informationen stammen. Diese Praxis dient dazu Fakten überprüfbar zu machen, wissenschaftliche Arbeiten zu würdigen und den Wissenschaftsbetrieb effizient zu gestalten \citep{Heesen.2014}. %% Buch 2014:57 zitieren

Das heißt, alle im Text verwendeten Quellen sind anzugeben. Innerhalb des Textes verwenden Sie die Kurzangabe in Form einer eckigen Klammer, nach dem jeweiligen Abschnitt \citep{Rettig.2017}.
%Bei Büchern muss zusätzlich die Seite angegeben werden. 
Zitieren Sie eine Quelle wörtlich, spricht man von einem direkten Zitat. Diese sind zur klaren Kennzeichnung in Anführungszeichen zu setzen und Zeichen für Zeichen zu übernehmen (inklusiver etwaiger Fehler). Erstreckt sich ein direktes Zitat über mehrere Zeilen ist es einzurücken. Im naturwissenschaftlichen Bereich sind direkte Zitate unüblich und zu vermeiden. 
  
Übernehmen Sie Idee, Fakten, Argumentationen etc. ist dies ein indirektes Zitat. Hier geben Sie am Ende des Gedankengangs die Quelle an \citep{Heesen.2014}.

Am Ende des Textes - im Literaturverzeichnis - sind die vollständigen bibliografischen Angaben notwendig. Diese müssen mindestens Autor*in, Titel, Erscheinungsort und Erscheinungsdatum enthalten. Beispielweise: 

\begin{quote}
Nachname, Vorname. Titel, Erscheinungsort, Erscheinungsjahr. \\
Adams, Douglas. The Hitchhiker's Guide To The Galaxy. New York, 1979.
\end{quote}

Bei Zeitschriften ist zusätzlich der Titel der Zeitschrift, der Band und die Seitenzahl des Artikels  anzugeben, auch hier ein Beispiel: 
\begin{quote}
Nachname, Vorname. Titel, in: Zeitschrift (Band), Erscheinungsjahr, Seiten. \\
Leibfried, Dietrich et. al. Creation of a six atom ''Schrödinger cat'' state, in: Nature (438) 2005, S. 639–642.
\end{quote}
Bei Internetquellen ist auch der Verfasser, der Titel, die URL und das Abrufdatum anzugeben: 
\begin{quote}
Nachname, Vorname. Titel, URL, abgerufen am dd.mm.yyyy.  \\
Maas, Steve et. al. FEBiO User’s Manual Version 2.8, www.help.febio.org/FEBio/FEBio\_um\_2\_8/, Abgerufen am 03.06.2020. 
\end{quote} 
Das Literaturverzeichnis ist entweder alphabetisch oder nach Auftreten der Werke im Text zu sortieren. 
Die obigen Beispiel dienen als Orientierung, daher müssen Sie nicht genau diese Formatvorlage verwenden. Verwenden Sie die in \LaTeX \ implementierten Pakete. Wichtig und entscheidend ist eine vollständige und nachvollziehbare Quellenangabe. 

Bevor Sie eine wissenschaftliche Arbeit abgeben stellen sie unbedingt sicher, dass Sie nicht gegen die wissenschaftliche Praxis und gegen die gültigen Plagiatsrichtlinien verstoßen haben. Verwenden sie dafür Plagiatssoftware wie zum Beispiel Turnitin auf Moodle (\href{https://moodle.umit.at/course/view.php?id=3993}{www.moodle.umit.at/course/view.php?id=3993}). 
